\documentclass{article}\usepackage[]{graphicx}\usepackage[]{color}

test
%% maxwidth is the original width if it is less than linewidth
%% otherwise use linewidth (to make sure the graphics do not exceed the margin)
\makeatletter
\def\maxwidth{ %
  \ifdim\Gin@nat@width>\linewidth
    \linewidth
  \else
    \Gin@nat@width
  \fi
}
\makeatother

\definecolor{fgcolor}{rgb}{0.345, 0.345, 0.345}
\newcommand{\hlnum}[1]{\textcolor[rgb]{0.686,0.059,0.569}{#1}}%
\newcommand{\hlstr}[1]{\textcolor[rgb]{0.192,0.494,0.8}{#1}}%
\newcommand{\hlcom}[1]{\textcolor[rgb]{0.678,0.584,0.686}{\textit{#1}}}%
\newcommand{\hlopt}[1]{\textcolor[rgb]{0,0,0}{#1}}%
\newcommand{\hlstd}[1]{\textcolor[rgb]{0.345,0.345,0.345}{#1}}%
\newcommand{\hlkwa}[1]{\textcolor[rgb]{0.161,0.373,0.58}{\textbf{#1}}}%
\newcommand{\hlkwb}[1]{\textcolor[rgb]{0.69,0.353,0.396}{#1}}%
\newcommand{\hlkwc}[1]{\textcolor[rgb]{0.333,0.667,0.333}{#1}}%
\newcommand{\hlkwd}[1]{\textcolor[rgb]{0.737,0.353,0.396}{\textbf{#1}}}%
\let\hlipl\hlkwb

\usepackage{framed}
\makeatletter
\newenvironment{kframe}{%
 \def\at@end@of@kframe{}%
 \ifinner\ifhmode%
  \def\at@end@of@kframe{\end{minipage}}%
  \begin{minipage}{\columnwidth}%
 \fi\fi%
 \def\FrameCommand##1{\hskip\@totalleftmargin \hskip-\fboxsep
 \colorbox{shadecolor}{##1}\hskip-\fboxsep
     % There is no \\@totalrightmargin, so:
     \hskip-\linewidth \hskip-\@totalleftmargin \hskip\columnwidth}%
 \MakeFramed {\advance\hsize-\width
   \@totalleftmargin\z@ \linewidth\hsize
   \@setminipage}}%
 {\par\unskip\endMakeFramed%
 \at@end@of@kframe}
\makeatother

\definecolor{shadecolor}{rgb}{.97, .97, .97}
\definecolor{messagecolor}{rgb}{0, 0, 0}
\definecolor{warningcolor}{rgb}{1, 0, 1}
\definecolor{errorcolor}{rgb}{1, 0, 0}
\newenvironment{knitrout}{}{} % an empty environment to be redefined in TeX

\usepackage{alltt}
\usepackage{hyperref}

\title{Deforestation and South East Asia}
\author{Madison Vorva and Marc Los Huertos}
\IfFileExists{upquote.sty}{\usepackage{upquote}}{}
\begin{document}

\maketitle

\section{Introduction}

\subsection{Objectives}

Each student will analyze different sections / regions of SE Asia and determine the amounts of deforestion, estimate Palm plantations area and fire frequency and area -- and calculate the carbon loss from the forests and habitat loss for endangered XX?

NOTES from Madi:


Global Forest Watch:
\url{http://www.globalforestwatch.org/}


Forest monitoring designed for action

\href{www.globalforestwatch.org}{Global Forest Watch} offers the latest data, technology and tools that empower people everywhere to better protect forests.


\subsection{Methods}

We (Madi? :-)) have created a grid of southeast Asia that we will sample from for the project. Each student will recieve XX random locations from the grid to analyze. 

Let's say there will be 1,000,000 grids, each each with about 5 square km to analyze. Students do 5 each. 5*5*15 students, 45 km2 that a sampling rate of 0.0075\%. That would be pretty respectable, for a class project!

Comparing LandSAT images? between 19XX and 2017, students will draw polygons around palm plantations and identify recently burned areas. Students will also measure the total land area in the grid. For each grid sample students will record information to fill out Table~\ref{tab:dataentry}.

Perhaps we compare what global forest watch has and what we come up?

\begin{table}[h]
\caption{Data Entry Table}
\label{tab:dataentry}
\begin{tabular}{|l|c|c|c|} \hline
GRID ID  & Area Analyze (km2)  &  Palm (km2) &  Recently Burned (km2) \\ \hline\hline
-9999     & XX    & XX  & XX  \\

\hline
\end{tabular}
\end{table}


\section{Results}

Students will combine their results to create a table that includes the following information:

\begin{table}[h]
\begin{tabular}{lcccc} \hline
Country   & Total Area (km2)  & Area Analyzed  & \% Palm & \% Recently Burned \\ \hline\hline

Cambodia & 181,035 &&& \\
Indonesia   & 1,904,569 & & & \\
Loas & 236,800 &&&\\
Malaysia   & 330,803 & & & \\
Myanmar & 	676,578 &&& \\
Philippines   & 300,000 & & & \\
Singapore & 	716 &&& \\
Thailand    & 513,120 &&& \\
Vietnam   & 331,212 & & & \\
\hline


\end{tabular}
\end{table}

using QGIS or ArcMap or ???

Madi suggests we use ArcGIS (ESRI's web based platform). \url{http://claremont.maps.arcgis.com/home/}

Question: Does CUC have student accounts for this?  Marc has emailed Warren to find out.


\end{document}
